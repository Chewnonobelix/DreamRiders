\part{Création de personnage}

Tous nouveaux personnage commence avec les caractéristiques suivante:
\begin{itemize}
\item[CC]
\item[CT]
\item[F]
\item[E]
\item[Ag]
\item[Int]
\item[Vol]
\item[Per]
\item[Soc]
\item[Vit]
\end{itemize}

Il choisit ensuite sa classe, le joueur applique alors les modifications de caractéristiques, de compétences, de talents et de traits indiquer. \\
Il choisit également sa classe, il applique alors les modifications indiqué par la classe. \\

Une fois toutes les modifications préalable fourni par la race et les classes effectué, le joueur peut personnalisé son personnage:
\begin{itemize}
\item [A réécrire]Il peut augmenter ou diminuer n'importe quelle caractéristique en diminuant ou augmentant n'importe qu'elle autre caractéristique du même nombre de points. (e.g: Jean veut augmenter la valeur de force de son personnage de 2 points, il doit alors enlever 1 point dans n'importe combinaison de caractéristique, ici il choisit d'enlever 1 point en volonté et 1 point en sociabilité)
\item [A réécrire] Il peut également baisser ses compétences jusqu'à un minimum de -2. Chaque niveau de compétence de base valent 1 point, et chaque niveau de compétence avancé valent 2 points. e.g: Pour augmenter une compétence avancé de 1 point, il faut soit baisser des compétences de base de 2 points, soit baissé une compétence avancé de 1 point.
\end{itemize}

Le joueur peut ensuite sélectionné son équipements dans la liste des équipements disponible pour ses classes.




\section{Expérience et niveau}
Un personnage peut dépenser son expérience acquise au cours de ses aventures pour acheter de nouveaux talents. \\
A chaque palier franchis, le personnage +E point de vie, 1 point de compétence, et tous les niveaux pair 1 point de caractéristique.