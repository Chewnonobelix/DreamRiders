\part{Règles}

\section{Caractéristiques principale}

Les \index{caractéristiques} sont au nombres de 10 et représente les capacité physique intrinsèques d'un personnage. Les caractéristiques ont une valeur comprise entre 0 et 9. Une caractéristique à 0 est une caractéristique inutilisable, ayant un effet différents en fonction de la caractéristique. \\
De plus chaque caractéristique aura un effet supplémentaire si elle se trouve dans les caractéristiques principale d'un rôle. Un rôle ne peut avoir plus de 2 caractéristique principales. Si un caractéristique principal se retrouve dans 2 rôle différents, elle ne se cumule pas.

\begin{itemize}
\item \index{capacite de combat@capacité de combat}
\item Capacité de tir
\item Force Quand la force est l'une des caractéristiques principale, le personnage ajoute 2 fois la force ses dégâts au corps à corps.
\item Endurance Gagne 2 fois l'endurance à chaque niveau.
\item Agilité Le personnage gagne 1 réaction supplémentaire.
\item Volonté Divise par 2 le nombre de PA nécessaire pour la magie.
\item Intelligence Double le nombre de connaissance maximum.
\item Sociabilité Affecte 2 fois plus de personne.
\item Perception Immunité à la surcharge sensorielle. 
\item Vitesse Le personnage double son nombre de PA.
\end{itemize} 

Un test de caractéristique se fait sur 1D10, le résultat - modificateur doit être inférieur ou égal à la caractéristique concerné. (e.g: Un personnage a une force à 3, pour réussir un test avec un modificateur à 0  il doit faire 3 ou moins sur 1D10, mais avec un modificateur de -1 il devrait faire 2 ou moins). Un résultat de 1 naturel est appelé réussite critique, et entraine une réussite automatique quelque soit le modificateur. Un résultat de 0 naturel est appelé échec critique et entraine un échec automatique quelque soit les modificateurs. Un résultat dit naturel est le résultat affiché par les dès sans tenir compte de quelconque modificateurs.\\
Le degrés de réussite (ou d'échec) correspond à la soustraction (seuil de réussite - résultat naturel + modificateur), un test est réussi si le degrés de réussite est supérieur ou égal à 0, sinon il est échoué.\\
Un test opposé est un test qui confronte 2 test, un pour chaque protagoniste. Le succès (ou non) d'un test opposé correspond au résultat de la soustraction des degrés de réussite, si le résultat est positif le test est un succès, sinon il est raté.

\subsection*{Limitations}
Certaines caractéristique entraine des limitations:
\begin{itemize}
\item Vous pouvez porté 20Kg/F
\item Vous connaitre avoir 1 tactique/Int
\item Vous pouvez connaitre 1 langue/Int
\item Vous pouvez avoir 3 connaissance/int (hors connaissance général)
\end{itemize}
\section{Caractéristiques annexe et connexe}
\subsection{Points De vie}
Le nombre de point de vie initial est de 1D10+E. \\
A chaque niveau le PJ gagne E point de vie 
\subsubsection{Dégât critique}
TODO
\subsection{Mort, KO et coma}

\subsubsection*{Mort}
TODO

\subsubsection*{Coma et perte de connaissance}
Un personnage tombant dans le coma ou perdant ne peux plus agir. \\
Un personnage peut perdre connaissance suite à une attaque spécial (e.g: étranglement, gaz somnifère, .... ). La durée d'une perte de connaissance est généralement  de quelques minutes, voir une poignée d'heure. Il n'est pas forcément nécessaire d'appliquer des soins pour faire reprendre connaissance à un personnage. \\
Un coma est une perte de connaissance beaucoup plus grave, pouvant durée plusieurs jours ou mois, voir de durée indéterminé. Dans un cas de coma des soins spécifique sont nécessaire pour limiter la durée du coma. Un personnage tombant peut en garder des séquelles. \\
Un personnage tombe automatiquement dans le coma si il subit un total de dégâts supérieur ou égal à 3 fois son endurance. Le personnage se réveillera naturellement au bout de (3*E) heures. 

\subsection{Initiative}
Le bonus de base à l'initiative est égal à la vitesse.

\subsection{Points d'action}
Le nombre de points d'action est déterminé par la formule suivante: vitesse / 2 (arrondis à l'inférieur) + 1. (1 points d'action pour une Vitesse de 0 ou 1, 2 pour une vitesse compris entre 2 et  3, ... \\
Les points sont nécessaire en situation de combat pour effectuer des actions, comme des mouvements, attaque ou lancer un sort par exemple. \\

\section{Test de compétence}
Une compétence est action particulière réalisable par un personnage. \\
Une compétence est associé à une caractéristique particulière. \\
Un test de compétence se comporte comme un test de caractéristique, auquel on apporte un modificateur spécifique.

\section{Point de destin}

Un point de destin peut être utilisé pour relancer une seule fois un test, dans ce cas le nouveau résultat est pris en compte quelque que soit le résultat. Les points de destin ainsi utilisé sont récupérer au bout d'un moment. \\
En cas de la mort d'un personnage, un point de destin peut être  définitivement retiré pour que le personnage revienne à la vie. \\ \\
Le nombre de points de destins initial est de 1D10/2 arrondis au supérieur.



\section{Santé mentale}
TODO
\section{Motivation}
TODO
\section{Artisanat}
TODO \\
Pour chaque fabriquer d'objet vous aurez accès un certain nombre de points utilisable. \\
Pour réussir la fabrication d'un objet, vous devez réussir le test d'artisanat correspondant au type de l'objet, avec une difficulté à +0. \\
Un test simplement réussi (degré de réussite à 0) vous donnera accès à X points de craft, +N points par degrés de réussite. \\
Un objet peut être améliorable. Pour ce faire vous devez faire le test d'artisanat correspondant qui vous donnera des points supplémentaire à dépenser (cf. paragraphe ci-dessus).
\subsubsection*{Arme}
\begin{itemize}
\item 1 point par dès de dégât
\item 1 point par modificateur de base
\item 1 point par pénétration d'armure
\item n points en fonction de l'attribut
\item 1 point par balle dans un mode
\item 1 point pour x mètre de portée en fonction de la catégorie de l'arme
\item Autonomie de base défini par la catégorie de l'arme. 1 point pour augmenter l'autonomie de 1
\end{itemize}
Rechargement: 1PA par tranche de 20 balles
\begin{itemize}
\item Redonne 1 point pour augmenter le rechargement d'un PA
\item Coûte 1 point pour diminuer le rechargement d'un PA
\end{itemize}

\subsection*{Armure}

\begin{itemize}
\item 1 point par PA
\item n points en fonction de l'attribut
\end{itemize}

\section{Combat}

Un combat est composé d'un certain nombre de tour. Chaque tour est composé d'un cycle effectué par chaque intervenant, PJ comme PNJ. \\
En préambule du combat chaque intervenant détermine son initiative, cet valeur restera la même pour tout le combat. L'initiative est égal à 1D10 + vitesse + modificateurs. \\
Par ordre d'initiative décroissant, i.e: de la plus haute valeur d'initiative à la plus basse: \\
\begin{enumerate}
\item Choisir les actions pour le tour. Le nombre d'actions d'attaque disponible par tour est de \textit{Vitesse/2(arrondis à l'inférieur) + 1}. Si une action nécéssite plus de point d'action que le montant actuel, le nombre de points nécessairerestant est reporté sur le(s) tour(s) suivant. Par exemple, si Paul à 2 points d'action, et qu'il veut faire une action coûtant 3 points, il utiliseras 2 points ce tour ci (en passant sont tour), utiliseras 1 point sur son tour suivant, et il lui restera 1 point à utiliser sur son tour.

\item Résoudre les actions
\end{enumerate} 

\subsection{Mouvements}
Durant un combat un personnage peut se déplacer. Il a a\index{capacite de combat@CC}ès à 4 types de déplacements: \textit{La marche}, \textit{la marche rapide}, \textit{la charge} et \textit{le sprint}, coutant respectivement 1,2,3 et 4 points d'action.
\begin{itemize}
\item[Marche] Le nombre de mètre maximal effectué par une marche simple est égal à la vitesse.
\item[Marche rapide] Le nombre de mètre maximal effectué par une marche rapide est égal au double de la vitesse.
\item[Charge] La charge est une action de mouvement permettant de se rapprocher rapidement d'un adversaire. Une charge s'effectue forcement vers un cible. Une charge offre également un bonus de +1 au test de \index{capacite de combat@CC} suivant effectuer dans le même tour. Un personnage chargeant doit au minimum se déplacer de la valeur de la vitesse, et au maximum 3 fois la vitesse.
\item[Sprint] Un sprint est un déplacement très rapide, d'un nombre de mètre maximal égal à 6 fois la vitesse.
\end{itemize}

\subsection{Attaque}
Une attaque consiste en test d'attaque, qui dépend du type d'arme utilisé, éventuellement annuler par une réaction, puis de la détermination des dégâts.
\subsubsection*{Corps à corps}
Une attaque au corps à corps est une attaque utilisant soit une arme de corps à corps, soit une partie de son corps. e.g: Ses poings, ses pieds ... \\
Un test d'attaque au corps à corps est un test de \index{capacité de combat@CC}, qui peut être paré ou esquiver (cf:\hyperlink{reac}{Réaction}). Si l'attaque n'est pas annulé, le joueur détermine les dégât en lançant le nombre de dés indiqué dans la description de l'arme, plus le modificateur indiquer (si aucun modificateur n'est indiquer, il est de +0), plus la force, plus d'autre modificateur éventuel. 

\subsubsection*{Distance}
Une attaque à distance est une attaque utilisant des armes à distance, naturelle ou non. \\
Un test d'attaque à distance est un test de CT, qui peut être paré ou esquivé. \\
Si l'attaque n'est pas annulé, le joueur lance le nombre de dés indiquer sur le profil de l'arme en y rajoutant d'éventuels modificateur.

\subsubsection*{Dégât}
Une fois les dégâts déterminés, le joueur ôte au total des dégât la classe d'armure correspond au type de dégât, et la valeur d'endurance de la cible. Si la valeur final est supérieur à 0, alors la cible enlève un nombre de point de vie égal à cette valeur final.

\hypertarget{reac}{\subsection{Réaction}}
A chaque tour, un personnage peut tenter un certain nombre de d'action de réaction pour contrer une ou plusieurs attaques. \\
Le nombre de réaction est égal au Vitesse / 2 arrondis au supérieur. \\
Les actions de réaction sont de 2 types:
\begin{enumerate}
\item Parade: Une parade consiste en un test de corps à corps opposé au test d'attaque adverse, dont le degrés de réussite détermine le nombre d'attaque paré.
\item Esquive: Une esquive consiste en un test d'agilité, dont le degrés de réussite détermine le nombre d'attaque esquiver.
\end{enumerate}


\subsection{Liste des action non exhaustive}

\begin{tabular}{|l|l|l|}
\hline
Nom de l'action & Point d'action & Description \\ \hline
Attaque simple (Coup par coup) & 1 & \\ \hline
Attaque semi-automatique & 2 & \\ \hline
Attaque automatique & 3 & \\ \hline
Visée & 1 & \\ \hline
Marche & 1 & \\ \hline
Marche rapide & 2 & \\ \hline
Charge & 3 & \\ \hline
Sprint & 4 & \\ \hline
Parade & 1R & \\ \hline
Esquive & 1R & \\ \hline
Feinte & 1 & \\ \hline
Incantation magique & Variable & \\ \hline
Rechargement & Variable & \\ \hline


\hline
\end{tabular}