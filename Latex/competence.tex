\part{Compétence}

Une compétence est un action réalisable par un personnage joueur ou non joueur. \\
Une compétence est basé sur une caractéristique particulière, et représente un cas spécifique d'utilisation d'une caractéristique. \\
Une compétence peut avoir un modificateur par défaut compris en tre -2 et +2. \\
Un test de compétence est un test sur 1D10, dont le résultat doit être inférieur ou égal à la valeur de la caractériqtique associé, plus ou moins un modificateur. \\

\subsection{Groupe de compétences}
Plusieurs compétences peuvent être regroupé au sein de ce qu'on appelle un groupe de compétence. Toutes les compétences au sein d'un même groupe de compétence fonctionnent de la même manière et sont associé à une même caractéristique. Chaque compétence doivent être apprise distinctement et indépendamment.

\section{Compétence de base}
Une compétence utilisable par défaut par personnage. Il n'est pas nécessaire qu'il l'ai appris en amont. \\
Une compétence est une action considéré comme admise par tout le monde. Par exemple parlé sa langue natale, ou tenter de barratiner un autre personnage. \\
La liste exacte et exhaustive des compétences de base est à définir par le maitre de jeu.
 
\subsection{Liste des compétence de base}

\section{Compétence avancé}
Une compétence avancé est une compétence dont il est nécessaire de faire l'acquisition préalable. \\
Un maitre de jeu peut, à loisir, rajouter des compétence avancé. \\ 

\subsection{Liste des compétence avancé}