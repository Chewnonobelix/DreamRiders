\part{Equipements}

\section{Armes}
Une arme est définie par un certains nombre d'attributs.
\begin{itemize}
\item[Nom] Le nom de l'arme
\item[Groupe]Le groupe l'arme. Le groupe de l'arme est dans la liste suivante: Corps à corps (CaC), Poing, base, lourde, montée, jets.
\item[Type] Type de dégât, le MJ peut rajouter des types de dégât si il le souhaite. La liste de base est: Physique, magique. Une arme peut avoir plusieurs types de dégât.
\item[Dégât] Le nombre de dégât infligé par type de dégât, sous la forme nD10 + x, avec n minimum à 1 et x minimum à 0. Il y a une valeur de dégât par type de dégât. 
\item[Pénétration d'armure] La pénétration d'armure pour le type de dégât correspondant. 
\item[Portée] La portée maximale de l'arme.
\item[Autonomie de tir] Le nombre de tir avant rechargement.
\item[Rechargement] Le nombre de points d'action nécessaire pour recharger l'arme
\item[Mode] Il s'agit des différents types d'action disponible pour l'arme. TODO
\item[Règles spéciale] L'ensemble des attributs spéciaux de l'arme
\end{itemize}
\subsection{Armes de corps à corps}
\subsection{Armes à distance}
\subsection{Munitions}
\section{Armures et protection}
\section{Outils}
\section{Matériaux}
