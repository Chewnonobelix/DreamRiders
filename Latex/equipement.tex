\part{Equipements}

\section{Armes}
\subsection{Armes à distance}
Une arme à distance est définie par un certains nombre d'attributs.
\subsubsection{Caractéristique}

\begin{itemize}
\item[Nom] Le nom de l'arme
\item[Groupe]Le groupe l'arme. Le groupe de l'arme est dans la liste suivante: Poing, base, lourde, montée, jets.
\item[Type] Type de dégât, le MJ peut rajouter des types de dégât si il le souhaite. La liste de base est: Physique, magique. Une arme peut avoir plusieurs types de dégât.
\item[Dégât] Le nombre de dégât infligé par type de dégât, sous la forme ND10 + X, avec n minimum à 1 et x minimum à 0. Il y a une valeur de dégât par type de dégât. 
\item[Pénétration d'armure] La pénétration d'armure pour le type de dégât correspondant. 
\item[Portée] La portée maximale de l'arme.
\item[Autonomie de tir] Le nombre de tir avant rechargement.
\item[Rechargement] Le nombre de points d'action nécessaire pour recharger l'arme
\item[Mode] Il s'agit des différents mode de tir pour l'arme concerné
\item[Règles spéciale] L'ensemble des attributs spéciaux de l'arme
\end{itemize}
\subsubsection{Munitions}

\subsection{Armes de corps à corps}
\subsubsection{Caractéristique}
\begin{itemize}
\item[Nom] Le nom de l'arme
\item[Type] Type de dégât, le MJ peut rajouter des types de dégât si il le souhaite. La liste de base est: Physique, magique. Une arme peut avoir plusieurs types de dégât.
\item[Dégât] Le nombre de dégât infligé par type de dégât, sous la forme ND10 + X, avec n minimum à 1 et x minimum à 0. Il y a une valeur de dégât par type de dégât. 
\item[Pénétration d'armure] La pénétration d'armure pour le type de dégât correspondant. 
\item[Règles spéciale] L'ensemble des attributs spéciaux de l'arme
\end{itemize}

\section{Armures et protection}
\subsection{Armures}
\subsubsection{Caractéristique}
\begin{itemize}
\item[Point d'armure]
\item[Type]
\item[Règles spéciale]
\end{itemize}
\subsubsection{Localisation}
Une armure est composé de plusieurs parties, chaque partie est assigné à une localisation du corps. \\
\begin{itemize}
\item[Tête]
\item[Corps]
\item[Bras]
\item[Jambe]
\item[Tous]
\end{itemize}
\subsection{Boucliers}
\subsubsection{Caractéristique}
\begin{itemize}
\item[Point d'armure]
\item[Type]
\item[Règles spéciale]
\end{itemize}
\section{Outils}
\section{Matériaux}
\section{Artisanat}