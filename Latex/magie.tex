\part{Magie}

\section{Utilisation}
Réussir un test de Volonté (moyen +0) et appliquer les effets décrits la description du sort. \\

	Vous pouvez surchargé un sort en augmentant votre niveau d'école. Pour chaque niveau de surcharge vous augmentez la difficulté du test de volonté, -1 par niveau supplémentaire (ex : si vous voulez augmenté le niveau de votre sort de 3, vous aurez un malus de -3 à votre test de volonté). De plus vous augmentez le nombre de points d'action nécessaire de 1 par tranche de 2 niveaux (e.g: +1 niveau  => +0 point d'action, +2 niveaux => +1 point d'action, +4 niveaux => +2 point d'action) \\
	Si vous augmentez le niveau de votre sort au-delà de 2, vous générerez un phénomène chaotique. \\
	Vous pouvez également sous-chargé un sort en réduisant votre niveau d'école, pour chaque niveau enlever vous aurez un bonus de +1 à votre test de volonté. Si vous diminuez votre sort au-delà de 2 (jusqu'à un minimum de 1), vous générerez un phénomène ordinal. De plus vous diminuez le nombre de points d'action de 1 par tranche de 2 niveaux, avec un minimum de 0 (e.g: -1 niveau => -0 point d'action, -2 niveaux => -1 point d'action, -4 niveaux => -2 point d'action). \\
	En surchargeant ou en sous-chargeant un sort, vous ajoutez ou enlevez N amélioration d'un sort au choix du PJ. \\
	
\section{Phénomène chaotique}
TODO
\begin{tabular}{|c|c|}
\hline
1 &  La végétation pourri sur 20M\\
\hline
2 &  La nuit tombe\\
\hline
3 &  L'eau se tarie\\
\hline
4 &  \\
\hline
5 &  Enragement de la cible pour 1 round\\
\hline
6 &  Enragement de toutes les créatures à 20m pour 1 round\\
\hline
7 &  \\
\hline
8 &  \\
\hline
9 &  \\
\hline
0 &  Un démon apparait\\
\hline
\end{tabular}

\section{Phénomène ordinal}
TODO
\begin{tabular}{|c|c|}
\hline
1 &  Un parterre de végétation apparait\\
\hline
2 & La luminosité augmente \\
\hline
3 &  Une rivière se met à couler\\
\hline
4 &  \\
\hline
5 &  Étourdissement de la cible sur 1 round\\
\hline
6 &  Étourdissement de toutes les créature à 20m sur 1 round\\
\hline
7 &  \\
\hline
8 &  \\
\hline
9 &  \\
\hline
0 & Un ange apparait \\
\hline
\end{tabular}

\section{Ecole de magie et catégorie de sort}
Chaque personnage possèdent un niveau associé à chaque école allant de 0 à 10.
Il existe au total 8 écoles de magie: \\
\begin{itemize}
\item Elémentaire \\
\begin{itemize}
\item Feu
\item Eau
\item Terre
\item Air
\end{itemize}
\item Divine \\
\begin{itemize}
\item Blanche
\item Noir
\end{itemize}
\item Biomagie \\
\begin{itemize}
\item Nature
\item Nécromancie
\end{itemize}
\item Psy\\
\begin{itemize}
\item Esprit
\item Relativiste
\end{itemize}
\end{itemize}

Les différent sorts sont répartis en plusieurs catégories:
\begin{itemize}
\item Pouvoir de manipulation, il s'agit de pouvoir basique (en général unique) permettant de manipuler le type de pouvoir de l'école. Les pouvoirs ne manipulation ne font aucun dégat et n'offre aucune protection, ils peuvent etre utiliser pour effectuer des représentation basique. Les pouvoir basique sont acquis automatiquement dès le 1er niveau d'une école atteint.
\item Pouvoir offensif, il s'agit de pouvoir servant généralement à provoquer des dégats dans les rangs adverses. 
\item Pouvoir défensif/utilitaire, il s'agit de pouvoir servant généralement à la protection, ou a une modification durable de la cible.
\item Pouvoir de buff/debuff, il s'agit de pouvoir servant à augmenter ou à réduire les capacité de la cible.
\item Pouvoir ultime, il s'agit d'une pouvoir unique, représentant la très haute maitrise d'une école.
\end{itemize}
\section{Sort}
%Présentation sort
% Nom:
% Ecole:
% Portée:
% Dégat:
% Point d'action:
% Durée:
% Description:



%Type de sort:
%Pour voir de manipulation (basique)
%Pouvoir offensif (upgradable)
%Pouvoir défensif (upgradable)
%Pouvoir ultime (unique)
%Pour voir de boost/debuff (upgradable)
\subsection{Ecole du feu}
\begin{itemize}
\item Boule de feu
\item Mur de feu
\item Hydre
\item Etincelle
\item Double dragon (sort ultime)
\item Contrôle
\item Invocation
\end{itemize}

\subsection{Ecole de l'eau}
\begin{itemize}
\item Déferlante
\item Solidification
\item Evaporation
\item Bouclier d'eau
\item Marée (sort ultime)
\item Invocation
\item Contrôle
\end{itemize}

\subsection{Ecole de l'air}
\begin{itemize}
\item Tornade
\item Eclair
\item Message
\item Bulle
\item Ouragan (sort ultime)
\item Invocation
\item Contrôle
\end{itemize}

\subsection{Ecole de la terre}
\begin{itemize}
\item Bouclier
\item Séisme
\item Pétrification
\item Force
\item Fortification (sort ultime)
\item Contrôle
\item Invocation
\end{itemize}

\subsection{Ecole blanche}
\begin{itemize}
\item Lumière
\item Révélation
\item Renforcement
\item Protection
\item Brèche angélique (sort ultime)
\end{itemize}

\subsection{Ecole noir}
\begin{itemize}
\item Affaiblissement
\item Corruption
\item Obscurité
\item Dissimulation
\item Erosion
\item Brèche démoniaque (sort ultime)
\end{itemize}

\subsection{Ecole de la nature}
\begin{itemize}
\item Soin
\item Régénération
\item Invocation animale
\item Contrôle de la flore
\item Résurrection (sort ultime)
\end{itemize}

\subsection{Ecole de la nécromancie}
\begin{itemize}
\item Poison
\item Flétrissement
\item Invocation de mort-vivant
\item Contrôle des morts-vivants
\item (sort ultime)
\end{itemize}

\subsection{Ecole de l'esprit}

\subsection{Ecole relativiste}